\documentclass{article}
\usepackage[utf8]{inputenc}
\usepackage[english]{babel}

\usepackage[backend=biber,%
            style=numeric-comp,     % how it appears in the literature overview
            citestyle=numeric-comp, % how it appears when citing in the text 
            maxnames=1]{biblatex}
\addbibresource{bibliography.bib} %Imports bibliography file

\usepackage{amsmath}
\usepackage{amssymb}
\usepackage{mathtools}
\mathtoolsset{showonlyrefs=true}  % only number equations that have been references in the text (set 'false' to number all)


\usepackage[framemethod=TikZ]{mdframed}
\usepackage{xcolor}


\usepackage{booktabs}
\usepackage{url} 
\usepackage{adjustbox}
\usepackage[ruled,linesnumbered]{algorithm2e}

\usepackage{pdflscape}
\usepackage{hyperref}
\usepackage{tabularx}

\frenchspacing % No double spacing between sentences
\linespread{1.2} % Set linespace
\usepackage[a4paper, lmargin=0.1666\paperwidth, rmargin=0.1666\paperwidth, tmargin=0.1111\paperheight, bmargin=0.1111\paperheight]{geometry} %margins


\newcommand{\st}{\text{s.t.}}
\newcommand{\defi}{:=}
\newcommand{\balpha}{\boldsymbol{\alpha}}
\newcommand{\bones}{\mathbf{1}}
\newcommand{\numexp}{n}
\newcommand{\numpair}{m}
\DeclareMathOperator{\sign}{sign}


\mdfdefinestyle{codeframe}{%
    linecolor=black,
    outerlinewidth=0.5pt,
    roundcorner=0pt,
    innertopmargin=\baselineskip,
    innerbottommargin=\baselineskip,
    innerrightmargin=20pt,
    innerleftmargin=20pt,
    backgroundcolor=gray!20!white}


\title{ROSVM Package - Mathematical Background}
\author{Eric Bach}

\begin{document}

\maketitle

\section{ToDo}

\begin{itemize}
    \item Add derivations for the exterior product features.
\end{itemize}

\section{Introduction}

This documents describes the mathematical background of the Ranking Support Vector Machine (RankSVM) \parencite{Joachims2002} implemented in the ROSVM package. 

\section{Method}

\subsection{Notation}
\begin{table}[t]
    \centering
    \caption{Notation table}
    \label{tab:notations}
    \begin{tabular}{ll}
        \toprule 
        {\bf Notation} & {\bf Description} \\ \midrule
        $\mathcal{P}$ & Set of preferences \\
        \bottomrule
    \end{tabular}
\end{table}

\subsection{Ranking Support Vector Machine (RankSVM)}

The RankSVM's primal optimization problem is given as: 
\begin{equation}
\begin{split}
    \underset{\mathbf{w},\boldsymbol{\xi}}{min} 
        &\quad f(\mathbf{w},\boldsymbol{\xi}) = \frac{1}{2}\|\mathbf{w}\|^2 + C\sum_{(i,j)\in P}\xi_{ij} \\
    \st &\quad y_{ij}\mathbf{w}^T(\phi_i-\phi_j)\geq 1-\xi_{ij},\quad\forall(i,j)\in\mathcal{P}\\
        &\quad \xi_{ij} \geq 0,\quad\forall(i,j)\in\mathcal{P},
    \label{eq:RankSVM_primal_problem}
\end{split}
\end{equation}
where $C>0$ is the regularization parameter. We define the pairwise labels as the retention time difference of the corresponding molecules, i.e. $y_{ij}\defi\sign(t_i-t_j)$. From the primal problem in Eq.~\eqref{eq:RankSVM_primal_problem} we can derive the following dual optimization problem:
\begin{equation}
\begin{split}
    \underset{\balpha}{\max} 
        &\quad g(\balpha) = \bones^T\balpha - \frac{1}{2} \balpha^T\left(\mathbf{y}\mathbf{y}^T\circ\mathbf{B}\mathbf{K}\mathbf{B}^T\right)\balpha \\ 
    \st &\quad 0\leq\alpha_{ij}\leq C,\quad\forall (i,j)\in \mathcal{P},
    \label{eq:RankSVM_dual_problem}
\end{split}
\end{equation}
where $\mathbf{y}\in\mathbb{R}^\numexp$ is the vector of pairwise labels, and $\mathbf{B}\in\{-1,0,1\}^{\numpair\times\numexp}$ with row $p=(i,j)$ being $[\mathbf{B}]_{p\cdot}=(0,\ldots,0,\underbrace{1}_{i},0,\ldots,0,\underbrace{-1}_{j},0,\ldots,0)$. For further details refer to the work by \cite{Kuo2014}. Using the properties of the Hadamard product $\circ$ we can reformulate the function $g(\balpha)$ of the problem in Eq.~\eqref{eq:RankSVM_dual_problem} \parencite{Styan1973}:
\begin{align}
    g(\balpha) 
        &= \bones^T\balpha-\frac{1}{2} \balpha^T\left(\mathbf{y}\mathbf{y}^T\circ\mathbf{B}\mathbf{K}\mathbf{B}^T\right)\balpha \\
        &= \bones^T\balpha-\frac{1}{2} \balpha^T\left(\mathbf{D}_\mathbf{y}\mathbf{B}\mathbf{K}\mathbf{B}^T\mathbf{D}_\mathbf{y}\right)\balpha \\
        &= \bones^T\balpha-\frac{1}{2} \balpha^T\mathbf{A}\mathbf{K}\mathbf{A}^T\balpha.
\end{align}
Here, $\mathbf{D}_\mathbf{y}\in\mathbb{R}^{\numpair\times\numpair}$ is a diagonal matrix storing the pairwise labels, and $\mathbf{A}\defi\mathbf{D}_\mathbf{y}\mathbf{B}\in\{-1,0,1\}^{\numpair\times\numexp}$. The matrix $\mathbf{A}$ now contains the pairwise labels as well by multiplying each row $p=(i,j)$ of $\mathbf{B}$ with $y_{ij}$, i.e. $[\mathbf{A}]_{p\cdot}=y_{ij}\cdot(0,\ldots,0,\underbrace{1}_{i},0,\ldots,0,\underbrace{-1}_{j},0,\ldots,0)$. 
\begin{mdframed}[style=codeframe]
    Check out '\texttt{\_build\_A\_matrix}' for the actual implementation of the $\mathbf{A}$-matrix construction from the data.
\end{mdframed}

\subsubsection{Optimizing the RankSVM Model Parameters}

% \begin{figure}[t]
%     \centering
\begin{algorithm}[t]
    \label{alg:frank-wolfe}
    \SetAlgoLined 
    \caption{Conditional gradient algorithm}
    Let $\balpha^{(0)}\in\mathcal{A}$\tcc*{A feasible initial dual variable.}
    \For{$k=0,\ldots,(K-1)$}{
        $\mathbf{s}\leftarrow\underset{\mathbf{s}'\in\mathcal{A}}{\arg\max}\,
            \left\langle\nabla g(\balpha^{(k)}),\mathbf{s}'\right\rangle$
            \tcc*{Solve sub-problem}
        $\gamma\leftarrow\frac{2}{k+2}$\tcc*{Step-size; also line-search possible}
        $\balpha^{(k+1)}\leftarrow(1-\gamma)\balpha^{(k)}+\gamma\mathbf{s}$\tcc*{Update}
    }
    $\balpha^*\leftarrow\balpha^{(K)}$\;
\end{algorithm}
% \end{figure}

We find the optimal RankSVM model $\balpha^*$ in the dual space given a training dataset $\mathcal{D}=\{(x_i,t_i)\}_{i=1}^\numexp$ using the conditional gradient algorithm \parencite{Jaggi2013}. The algorithm is shown in \ref{alg:frank-wolfe}. The feasible set is defined as $\mathcal{A}\defi\{\balpha\in\mathbb{R}^\numpair\,|\,0\leq\alpha_{ij}\leq C,\forall (i,j)\in\mathcal{P}\}$ which follows from the constraints of the dual optimization problem in Eq.~\eqref{eq:RankSVM_dual_problem}. Note that $\mathcal{A}$ is compact convex set.
\begin{mdframed}[style=codeframe]
    The function '\texttt{\_assert\_is\_feasible}' implements the feasibility check for a given $\balpha^{(k)}$ iterate.
\end{mdframed}

\paragraph{Solving the Sub-problem:}
In each iteration of Algorithm~\ref{alg:frank-wolfe} we need to solve the following linear optimization problem:
\begin{align}
    \mathbf{s}
        &=\underset{\mathbf{s}'\in\mathcal{A}}{\arg\max}\,\left\langle\nabla g(\balpha^{(k)}),\mathbf{s}'\right\rangle\\
        &=\underset{\mathbf{s}'\in\mathcal{A}}{\arg\max}\,\left\langle\underbrace{\bones-\mathbf{A}\mathbf{K}\mathbf{A}^T\balpha^{(k)}}_{\defi\mathbf{d}},\mathbf{s}'\right\rangle.\label{eq:sub_problem}
\end{align}
Eq.~\eqref{eq:sub_problem} can be solved by simply evaluating $\mathbf{d}$ and subsequently setting the components of $s\in\mathbb{R}^{\numpair}$ as:
\begin{equation}
    s_{ij}=\begin{cases}
            C&\text{if }d_{ij}>0\\
            0&\text{else}.
           \end{cases}
\end{equation}
\begin{mdframed}[style=codeframe]
    The function '\texttt{\_solve\_sub\_problem}' implements the sub problem solver.
\end{mdframed}


% \section{Notation}
% \begin{itemize}
%     \item $m_i\in\molspace$ molecule and $t_i\in\mathbb{R}$ its retention time
%     \item $\mathbf{x}_i\in\mathbb{R}^d$ representation of the molecule $m_i$ with dimension $d$
%     \item $\phi(\mathbf{x}_i)=\phi_i$ is the \emph{feature} representation of a molecule $m_i$
%     \item $\phi:\mathbb{R}^d\mapsto\mathcal{F}_x$ feature map 
%     \item $\{(\mathbf{x}_i,t_i)\}_{i=1}^{\ell}$ training dataset of (molecule representation, retention-time)-tuples
%     \item We say $m_i$ elutes before $m_j$ ($m_i\elutesbefore m_j$) if $t_i < t_j$.
%     \item $\Phi\in\mathbb{R}^{\ell\times|\mathcal{F}_x|}$, matrix storing the molecules' feature representations $\phi_i$ row-wise
% \end{itemize}
% 
% \section{Conditional gradient descent}
% 
% I am using the conditional gradient algorithm\cite{Ouyang2010,Jaggi2013} to train the Ranking Support-Vector-Machine (RankSVM) using its dual representation. The algorithm is shown in Table~\ref{alg:conditional-gradient}.
% 
% \begin{table}[h]
%     \centering
%     \caption{Conditional Gradient or Frank-Wolfe algorithm. We consider maximizing the function $g$ here, so $\text{argmin}$ becomes $\text{argmax}$ here.}
%     \label{alg:conditional-gradient}
% \begin{tabular}{ll}
%     \toprule
%     \multicolumn{2}{l}{\textbf{Input:} $\balpha^{(0)}\in\domain$ (solution in feasible set / domain), $k=0$}\\ \midrule
%     \textbf{repeat} & \\
%     1. & $\mathbf{s} = \underset{\mathbf{s}\in\domain}{\text{argmax}}\,\nabla g(\balpha^{(k)})^T\mathbf{s}$ \\
%     2. & Determine a step size $t$ for iteration $k$ \\
%     3. & \textbf{quit} if $t\leq 0$ \\
%     4. & Update $\balpha^{(k+1)}=(1-t)\balpha^{(k)}+t\mathbf{s}$ \\
%     \textbf{until} & stopping criterion is satisfied \\
%     \bottomrule
% \end{tabular}
% \end{table}
% 
% 
% \section{RankSVM with Slack Variable for each Training Pair}
% 
% \subsection{Optimization problem}
% 
% \paragraph{Primal}
% 
% The rankSVM is expressed by the following optimization primal problem:
% 
% \begin{align}
%     \underset{\mathbf{w},\mathbf{\xi}}{min} &\quad \frac{1}{2}\|\mathbf{w}\|_2^2 + C\sum_{(i,j)\in P}\xi_{ij} \\
%     \st &\quad \mathbf{w}^T(\phi_j-\phi_i)\geq 1-\xi_{ij}, \forall(i,j)\in\pref\label{prob:rankSVM-primal}\\
%         &\quad \xi_{ij} \geq 0, \forall(i,j)\in\pref,
% \end{align}                
% 
% where $C>0$ is the regularization parameter, and $\pref=\{(i,j)\,|\,m_i\elutesbefore m_j\}$. The optimization problem \eqref{prob:rankSVM-primal} learns a model $\mathbf{w}$ such that:
%     
% \begin{align}
%                     &\quad\mathbf{w}^T(\phi_j-\phi_i) > 0 \\
%     \Leftrightarrow &\quad\mathbf{w}^T\phi_j - \mathbf{w}^T\phi_i > 0 \\ \label{eq:pairwise_ordering_function}
%     \Leftrightarrow &\quad\mathbf{w}^T\phi_j > \mathbf{w}^T\phi_i ,\quad\text{ if } (i,j)\in\pref 
% \end{align}      
% 
% and therefore induces a general ordering of the molecules (based on their representation) that is correlated with their retention time.
%                     
% \paragraph{Dual}                    
%                 
%     The primal optimization problem \eqref{prob:rankSVM-primal} can be equivalently expressed as:
% \begin{align}
%     \underset{\mathbf{w},\mathbf{\xi}}{min} &\quad \frac{1}{2}\|\mathbf{w}\|_2^2 + C\sum_{(i,j)\in P}\xi_{ij} \\
%     \st &\quad 1-\xi_{ij} - \mathbf{w}^T(\phi_j-\phi_i) \leq 0, \forall(i,j)\in \pref\label{prob:rankSVM-primal-equiv}\\
%         &\quad -\xi_{ij} \leq 0, \forall(i,j)\in P.
% \end{align}
% 
% To find the dual problem of \eqref{prob:rankSVM-primal} we define the Lagrangian of problem \eqref{prob:rankSVM-primal-equiv}:
% 
% \begin{align}
%     L(\mathbf{w},\mathbf{\xi},\mathbf{\alpha},\mathbf{\beta}) 
%         &= \frac{1}{2}\|\mathbf{w}\|_2^2+C\bone^T\mathbf{\xi}+\left(\sum_{(i,j)\in\pref}\alpha_{ij}(1-\xi_{ij}-\mathbf{w}^T(\phi_j-\phi_i))\right)-\mathbf{\beta}^T\mathbf{\xi}\\
%         &= \frac{1}{2}\|\mathbf{w}\|_2^2+C\bone^T\mathbf{\xi}+\balpha^T\bone-\balpha^T\mathbf{\xi}-\left(\sum_{(i,j)\in\pref}\alpha_{ij}\mathbf{w}^T(\phi_j-\phi_i)\right)-\mathbf{\beta}^T\mathbf{\xi}\label{eq:rankSVM-lagrangian}\\
%         &= \frac{1}{2}\|\mathbf{w}\|_2^2+C\bone^T\mathbf{\xi}+\balpha^T\bone-\balpha^T\mathbf{\xi}-\balpha^T\mathbf{A}\Phi\mathbf{w}-\mathbf{\beta}^T\mathbf{\xi}.
% \end{align}
% 
% For the definition of the $\mathbf{A}\in\{-1,0,1\}^{|\pref|\times\ell}$ matrix see \parencite{Bach2018}. Now we get the derivatives of $L$ with respect to $\mathbf{w}$ and $\mathbf{\xi}$\footnote{Check out this great tool: \url{http://www.matrixcalculus.org/}}:
% 
% \begin{align}
%     \nabla_{\mathbf{w}}L(\mathbf{w},\mathbf{\xi},\balpha,\mathbf{\beta})&=\mathbf{w}-\Phi^T\mathbf{A}^T\balpha=0\Rightarrow\boxed{\mathbf{w}^*=\Phi^T\mathbf{A}^T\balpha}\\
%     \nabla_{\mathbf{\xi}}L(\mathbf{w},\mathbf{\xi},\balpha,\mathbf{\beta})&=C\bone-\balpha-\mathbf{\beta}=0\Rightarrow \boxed{C\bone=\balpha+\mathbf{\beta}}
% \end{align}
% 
% When we plug these solutions back into the Lagrangian (Equation~\eqref{eq:rankSVM-lagrangian}), we get the corresponding dual function problem of \eqref{prob:rankSVM-primal-equiv}:
% 
% \begin{align}
%     L(\mathbf{w}^*,\mathbf{\xi}^*,\balpha,\mathbf{\beta})
%         &= \frac{1}{2} \balpha^T\mathbf{A}\Phi\Phi^T\mathbf{A}^T\balpha - \balpha^T\mathbf{A}\Phi\Phi^T\mathbf{A}^T\balpha + \bone^T\balpha \\
%         &= - \frac{1}{2} \balpha^T\mathbf{A}\Phi\Phi^T\mathbf{A}^T\balpha + \bone^T\balpha \\
%         &= \boxed{- \frac{1}{2} \balpha^T\mathbf{A}\mathbf{K}\mathbf{A}^T\balpha + \bone^T\balpha = g(\balpha)}
% \end{align}
% 
% with the following constraints on the dual variables: $0\leq\alpha_{ij}\leq C,\,\forall (i,j)\in\pref$, and $\mathbf{K}=\Phi\Phi^T\in\mathbb{R}^{\ell\times\ell}$ being the training kernel matrix with $[\mathbf{K}]_{st}=\phi_s^T\phi_t$. Therefore the dual optimization problem of \eqref{prob:rankSVM-primal-equiv} is given as\footnote{If we minimize the primal than we maximize the dual objective}:
% 
% \begin{align}
%         \underset{\balpha}{\max} &\quad g(\balpha) = - \frac{1}{2} \balpha^T\mathbf{A}\mathbf{K}\mathbf{A}^T\balpha + \bone^T\balpha \label{prob:rankSVM-dual}\\ 
%         \st &\quad 0\leq\alpha_{ij}\leq C,\,\forall (i,j)\in \pref.
% \end{align}
% 
% \subsection{Optimization}
% 
% Lets consider the first step of the Conditional gradient algorithm.
% 
% \paragraph{1. Step: Finding $\mathbf{s}$}                
% To run the algorithm we need to find the gradient of $g(\balpha)$:
% 
% \begin{align}
%     \underset{\balpha}{\nabla}g(\balpha)=\underset{\balpha}{\nabla}\left(- \frac{1}{2} \balpha^T\mathbf{A}\mathbf{K}\mathbf{A}^T\balpha + \bone^T\balpha\right)=\bone - \mathbf{A}\mathbf{K}\mathbf{A}^T\balpha
% \end{align}
% 
% This gradient can be used to solve the following optimization problem:
% 
% \begin{align}
%     \underset{\mathbf{s}}{\max} &\quad  \underset{\balpha}{\nabla}g(\balpha)^T\mathbf{s} = (\bone - \mathbf{A}\mathbf{K}\mathbf{A}^T\balpha)^T\mathbf{s}\\
%     \st &\quad 0\leq\mathbf{s}\leq C.    
% \end{align}
% 
% If we define $\mathbf{d}=(\bone-\mathbf{A}\mathbf{K}\mathbf{A}^T\balpha)$ than the problem writes as:
% 
% \begin{align}
%     \underset{\mathbf{s}}{\max} &\quad \mathbf{d}^T\mathbf{s}\\
%     \st &\quad 0\leq\mathbf{s} \leq C.
% \end{align}                
% 
% The optimal $\mathbf{s}$ can be easily find by inspecting the signs of the $\mathbf{d}$ vector:
% 
% \begin{equation}
%     \mathbf{s}_{ij}  =\begin{cases}
%                       C & \text{if } d_{ij} > 0 \\
%                       0 & \text{otherwise}
%                    \end{cases}.
% \end{equation}
% 
% The proof idea here: The sum of the entries of $\mathbf{d}$ should be maximized. As $0\leq\mathbf{s}\leq C$ the largest coefficient should be given to all positive entries of $\mathbf{d}$, which is $C$. On the other hand, negative entries of $\mathbf{d}$ should not count into the sum and therefore get the multiplier $0$.      

\newpage
\printbibliography

\end{document}
